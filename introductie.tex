
In the digital world of today, malware is still a massive and growing problem. While back in the day it was used to annoy users and system administrators, nowadays it's used for extortion, cyber espionage and surveillance by criminal groups and rivalling governments. One of the main risk factors to get infected with malware is a drive-by download while visiting a normal day-to-day website because, for example, the website got hacked and infected. 

In many cases\footnote{http://www.proofpoint.com/threatinsight/posts/malware-in-ad-networks-infects-visitors-and-jeopardizes-brands.php} \footnote{http://blog.fox-it.com/2013/08/01/analysis-of-malicious-advertisements-on-telegraaf-nl} \footnote{http://blog.fox-it.com/2014/01/03/malicious-advertisements-served-via-yahoo}, it was not the actual website but one of the advertisement networks that was infiltrated and which subsequently started serving malicious code hidden in innocently looking advertisement code. This is also called malvertising \cite{Li2012}.

National CERT organisations are (of course) interested in an early detection of such threats. While automated systems to scan websites already exist, like Cuckoo\footnote{http://cuckoosandbox.org} and Anubis\footnote{http://anubis.iseclab.org}, one of the main difficulties is the time needed to analyse a single website and the maintenance needed to keep these systems up-to-date for the latest threats.

% Zeggen dat _wij_ dit en dat gaan doen om die en dat redenen

\section{Scope}

% Aanpassen aangezien we weten wat we hebben gedaan
% Mergen in introductie sectie
The scope of this project consists of creating an algorithm that allows multiple URLs to be opened at the same time while still being able to track all further interaction, such as unexpected HTTP requests and other malicious activity, and link them to the original request/URL. To prove that the algorithm is something feasible, we will also implement a proof of concept of the algorithm.

The goal during this research project is to make the algorithm platform agnostic. If this is not possible then we will limit our self to Windows 7 with version 8 of the Internet Explorer browser.

The detection and identification of malicious behaviour is not part of this project. For our PoC we will stick to the detection of a well-known older and still to be determined malware family which existence is easy to detect on the system. 

