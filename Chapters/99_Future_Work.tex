Our study focused on finding a generic algorithm which allows for large-scale detection of drive-by downloads. Currently, the analysis phase is done after processing the events and creating the graph. Real-time analysis on the graph would allow for faster detection and reporting but might have more overhead. The optimal moment to analyse the graph should be investigated.

During the proof of concept, an abstraction was made from low-level API calls to high-level events. While this greatly reduces the effort needed to analyse the graph, this poses a risk that crucial information might be missed. Further effort should be invested in finding an optimal granularity for the graph.

Additional intellectual effort should also be invested in the defining of relations between events. In our proof of concept, the relations between events are simplistic and straightforward. With more effort, better and more complex relations can be found and defined in the graph.

% Alles over de implementatie van het algoritme boeit niet, dat is niet waarover onze paper gaat
% detecting malware/drive-by downloads is dus niet echt future work
\subsection{Future work PoC}

The proof of concept in its current state is not ready for deployment. While tests during the development with real malware suggest that even a simplistic analyser is able to detect certain malware families, more advanced analysers should be developed. The current analyser also gives a false positive on a website that uses Java applets. While none of the tested websites use such applets, creating a whitelist of processes that can legitimately be started by browser plug-ins should be considered.

Another limitation is the stability of the proof of concept. In up to twenty percent of the cases, the proof of concept would never go to a completed state. As this was not related to a specific website or malware sample, this is not seen as a fundamental mistake in the algorithm, but a bug in the implementation. No time was available to resolve this issue and a major update for Cuckoo is around the corner that resolves several known issues.
