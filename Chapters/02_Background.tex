For a better insight into the project, it is necessary to introduce certain theoretical concepts first. In the next chapter this theory will be used to base the approach on for the design and development of the algorithm and choices regarding the proof of concept.

\subsection{Drive-by downloads}

Browsing the internet with an unpatched system, for example because the patch is not installed or there is simply no solution available, can be dangerous. All software contains mistakes and such mistakes are patched almost on a daily basis. Part of those mistakes can be (ab)used to get control over a computer system and be used to run malicious software without the user noticing.

If a website uses such mistake to take control over the web browser and download malicious software to the system, this is called a drive-by download\cite{Le2013}. Figure \ref{fig:dbdownload} shows the steps involved from visiting a website until the moment the system is infected with malware.

By compromising the web browser and injecting malicious code, the malware gets full control over the infected process, running with the same privileges as the web browser on the host system. Depending on the system configuration this either means that full system access is available or that additional steps are required to escalate the privileges to the intended level.


\begin{figure}[h]
    \centering
    \includegraphics[width=12cm]{Images/drive-by-download.png}
    \caption{The anatomy of a drive-by download malware infection. \cite{dbdownload-anatomy}}
    \label{fig:dbdownload}
\end{figure}

\subsubsection{Behaviour}
\label{sec:behavior}

What happens after a malware infection depends on the malware and the goals of the attacker. But usually the goals of the attacker are persistence and further exploitation. The former is achieved by writing executable code to the disk and optionally changing several configuration files to make sure the code is executed at certain events, for example after reboot. The latter is achieved by dowloading further malicious components. We thus expect to see extra network traffic after the first exploit.

While theoretically malware could directly communicate with the kernel, most malware\todo{need citation} behaves like a normal application and uses the installed, or with the operating system provided, libraries. The usage of such libraries can be detected when the access to them is monitored.

\subsection{API hooking}

In the early days, libraries were primarily used by an application by statically linking to it. This means that the library becomes part of the application and it is no longer possible to determine which part of the application was originally part of the used libraries.

For performance and maintainability, dynamic linking was invented. The application describes which libraries it needs and the linker of the operating system will glue the applications and its dependent libraries together in the memory space of the application. 

Because the linker has to know all exported functions and where in the library it can find such function, a symbol table is part of every dynamic library. The same information can be used to hook into an provided function during runtime or to trick the linker to load a replacement of a certain function because the modified version has a higher priority.

This is called API hooking\cite{}\todo{} and it is a very useful technique to monitor the behaviour of applications. The original function is replaced by a substitute. This substitute function calls, for example, the original function and logs the performed operation. Or the substitute could be a custom replacement of the original function.

The technical implementation of API hooking is highly complex and platform specific. Many different techniques\cite{jbremer2012} of hooking are possible as well. If the start of the application can be controlled, the linker search path can be extended to include the replacement library. Alternatively, the import section of the application can be modified. When the application is already loaded or modification of the application or system is unwanted, the function to hook can be overwritten in memory with a replacement or jump to a different location in memory. However, this will prevent the ability to execute the original function unless the overwritten bytes are carefully preserved and reconstructed somewhere else.

In this project API hooking will be used to reverse engineer the internal workings and API usage of web browsers and to log the behaviour of the web browser and malware for later analysis.

\subsection{Web browser architecture}
% Target: 4-6 blz

\todo{Meer bronnen?}

Modern web browsers are complex applications consisting of many components that have to work together. To develop a generic algorithm it is crucial to have first an in-depth insight in the working of the engines used to drive a web browser. In this project is focused on Internet Explorer, Mozilla Firefox, Chromium and Apple Safari. Those four browsers combined have a marketshare of more then 90\%\footnote{http://gs.statcounter.com/\#desktop-browser-ww-monthly-201412-201412-bar} \footnote{http://www.netmarketshare.com/browser-market-share.aspx?qprid=1\&qpcustomb=0}.

All modern web browsers allow the usage of multiple tabs with web pages in a single window. The underlying implementations differ creatly. In some browsers are only libraries provided by the operating system used while others decided to use their own. Also have some vendors decided to use multiple processes, sometimes even a new process for every single tab.

\textbf{Internet Explorer} is the well-known web browser from Microsoft. While until quite some time ago also a Mac version was available, in the last decade only the Windows version has been updated. Since version 7 is tabbed browsing available and version 8 was improved with the ability to run tabs from their own process (see figure \ref{fig:ie8proc}). This feature is called Loosely-Coupled IE\cite{http://blogs.msdn.com/b/ie/archive/2008/03/11/ie8-and-loosely-coupled-ie-lcie.aspx}. Every process runs independent from the other processes and runs its own network stack and instances of content plugins like Flash or Silverlight.

Starting each tab in its own process comes with an inevitable overhead by using more memory and the time it cost to start the process. For this reason can a process host multiple tabs. The amount of tabs to host in a single process and the maximum of processes to use is determined by the configuration. For backwards compatibility reasons is also the option provided to disable the usage of multiple processes and host all tabs in a single browser process.

The network stack used in Internet Explorer is provided by the Windows operating system and is called WinINet. This library provides high-level access to functions that allows applications to perform HTTP and FTP requests and utility functions for caching, proxys and security. After setting up the library and initiating and configuring the request, WinINet will perform the necessary steps to execute the request. WinINet depends for this on the Winsock library to setup the required network connections and Schannel to provide transparent support for SSL/TLS connections.

\begin{figure}
    \centering
    \includegraphics[width=9cm]{Images/IE8_process_model.png}
    \caption{The Internet Explorer process model starting from version 8. \cite{zelfde msdn link, modified}}
    \label{fig:ie8proc}
\end{figure}

\textbf{Firefox} is a cross-platform browser developed by the Mozilla foundation and one of the first that supported tabbed browsing. While it was first the main competitor for Internet Explorer, the release of Google Chrome made it lose some of its popularity.

In Firefox run only the plugins from a different process. The rendering of the web pages still happens from a single process. A long-term project to change that called Electrolysis\footnote{https://wiki.mozilla.org/Electrolysis} has been going on since 2009. In the new architecture is the entire rendering moved to a dedicated and sandboxed ``content'' process and is the main process used to host the user interface and serve as a proxy between the outside world and the content process. A longer-term goal is to spread the rendering of multiple tabs over more then one content process so that if a content process crashes not all tabs are affected.

To be platform independent is not directly interfaced with the provided libraries of the operating system. Instead a platform-neutral API called NSPR is used. Together with the NSS library that provides the functionality to create SSL/TLS connections is this used by the high-level network library called Necko. Necko provides the interface to perform HTTP and other protocol requests without revealing protocol, transport level or platform specific implementation details and is comparable to WinINet.

\textbf{Chromium} is the open-source version of the Google Chrome browser and is except a couple of proprietary components identical. While it's relative young, it is one of the most used web browsers. The big innovation of Chrome \cite{http://blog.chromium.org/2008/09/multi-process-architecture.html} was to use multiple processes instead of a single process. Besides its own process for every tab, have also the plugins and audio subsystem their own process. The subprocesses run in a sandbox with limited privileges and use the main process to communicate with the outside world.

\begin{figure}[h]
    \centering
    \includegraphics[width=9cm]{Images/Chrome_network.png}
    \caption{A high-level overview of the components involved in requesting an URL in Chromium. Platform specifics details related to sockets are hidden in StreamSocket and the usage of SSL/TLS is made transparent by using the interface-compatible SSLSocket instead of StreamSocket. \cite{http://www.chromium.org/developers/design-documents/network-stack, modified}}
    \label{fig:chrome_network}
\end{figure}

The library used by Chromium for network access is custom developed and tightly integrated in the engine. It provides similarly to Necko and WinINet a high-level interface (see figure \ref{fig:chrome_network}) but also it does contain lower level interfaces that interface directly with the operating system socket API. To provide transparent SSL/TLS support is the same library used as in Firefox, NSS.

\textbf{Safari} is the last browser that was examined in this project. It's the proprietary web browser of Apple that is since three years only available for their own operating system, Mac OS X. Around the same time\footnote{https://lists.webkit.org/pipermail/webkit-help/2011-July/002298.html} has support for using multiple processes been added. Safari is closed-source but build on top of many open-source components like JavaScriptCore and WebKit.

Safari uses until a certain limit a dedicated process for every tab. However for performance reasons and resource restrictions is after reaching the limit multiple tabs hosted in a single process. The network operations for the main and tab processes is concentrated in a dedicated process. Only this process will retrieve the webpages and use IPC mechanics to deliver the result to the correct process. 

CFNetwork is the library that is used by Safari for its network access. This library is one of the core frameworks of the OS X operating system and available for all applications. It provides interfaces for all relevant web related protocol. An unified interface called NSUrl like the other browsers use is also available, however because of the closed-source nature of Safari it's not possible without an extensive reverse engineering effort to determine if it's used instead of directly using the provided APIs of the CFNetwork library.

\iffalse

Which techniques are used by browsers to make concurrently visiting multiple
URLs possible?
	- Tabs of course
		- As a process
		- As multiple threads under the browser process

How can we link an HTTP request to its source URL without the modification of the used web browser?
we need extra information, see vraag 4
network is niet genoeg blablabla

How do web browsers make HTTP requests and retrieve webpages? Which Operating System level APIs are used?
	- Safari: CFNetwork and NSURLConnection(Loader) and IPC
		1 hoofdprocess: Safari
		1 process per tab: Safari Web Content
		1 process voor networking: Safari Networking
		Network stack of OS X.
	- Internet Explorer: C API calls naar Windows libraries
		Process per tab: is configureerbaar in settings/registry, je kan ook meerdere tabs in 1 process hebben
		Network stack of Windows
	- Firefox: C++ Library calls
		Single Process (+ 1 process voor flash)
		Own network stack Necko (nss3 voor trafiek te encrypten)
		\url{https://developer.mozilla.org/en-US/Firefox/Releases/3.5/Updating_extensions#Getting_a_load_context_from_a_request}
		http://stackoverflow.com/questions/10719606/is-it-possible-to-know-the-target-domwindow-for-an-httprequest
	- Chrome: IPC 
		1 hoofdprocess: Google Chrome die networking doet
		1 process per tab. (Altijd 13 threads?)
		1 process voor Flash.
		1 process voor Audio. (4 threads)
		Own network stack (nss3 voor trafiek te encrypten)
		http://www.chromium.org/developers/design-documents/network-stack

What extra information from the client's (running) machine can be used to augment the information gained from network trac to make the tracking of malware to its source URL easier?  
	- The Thread ID or Process ID van tabs, PDF reader, Java applet, ...
	- Handle bij IE
	- File descriptors
	- Process tree
	- Voordeel van op machine network traffic te intercepten is dat we rommel van andere applicaties niet zien, maar enkel het trafiek van de browser en de gespawnde subprocessen ervan.
	- 
\fi

\subsection{Correlating HTTP requests}

The challenge faced when multiple websites have to be loaded at the same time, is to known to which website a HTTP request corresponds. Many websites consists of dozens of resources that has to be loaded. The loading of some resources can even be delayed until after certain predefined events. In some libraries consists the requesting of a web resource of several independant steps.

An easy solution would be to modify the web browser in such way that it exposes this information with an easy to use interface. While this would solve the problem, regular maintenace would be required to keep this system up-to-date. A better solution would be, to be able to correlate a HTTP request to it's originating webpage by observing the behaviour and environment of the web browser.

A solution could be to log all the network traffic and analyse this. While this information is always available, it would require complex protocol and content parsers to reconstruct the original network streams and extract useful information from it. Loading resources over an encrypted connection would be impossible to notice, unless a proxy is used that would on-the-fly replace the certificate. Even then it would be trivial to circumvent this system as scripting languages and browser plugins could be used to dynamicly request resources.

Instead, by logging the API calls that are made by the web browser, the full process and thread context of every call is available or can be reconstructed from earlier calls. As all modern web browsers use high-level network libraries, this is the ideal place to monitor. When combined with other interesting APIs, a full insight in the behaviour of the web browser is available and detecting malicious behaviour is a matter of writing the correct behavioural analysers.