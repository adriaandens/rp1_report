% Target: 4-6 blz

Modern web browsers are complex applications consisting of many components which have to work together. To develop a generic algorithm, it is crucial to have an in-depth understanding of the inner workings of a web browser. This project focuses on Internet Explorer, Mozilla Firefox, Chromium and Apple Safari. Those four browsers combined have a marketshare of more than 90\%\footnote{http://gs.statcounter.com/\#desktop-browser-ww-monthly-201412-201412-bar} \footnote{http://www.netmarketshare.com/browser-market-share.aspx?qprid=1\&qpcustomb=0}.

All modern web browsers allow the usage of multiple tabs in a single window. The underlying implementations of those tabs differ greatly. Internet Explorer and Safari use only libraries provided by the operating system while Firefox and Chromium decided to use their own libraries. Some browsers decided to use multiple processes and sometimes even a new process for every single tab.

\textbf{Internet Explorer} supports tabs since version 7 and version 8 was improved with the ability to run tabs in their own process (see figure \ref{fig:ie8proc}). This feature is called ``loosely-coupled IE'' \cite{IE8LCIE}. Every process runs independently from the other processes and runs with its own network stack and instances of content plug-ins like Flash or Silverlight.

Starting each tab in its own process comes with an inevitable overhead of using more memory and a slower startup. For this reason a process in Internet Explorer can host multiple tabs. The number of tabs in a single process and the maximum number of processes is determined by the configuration. For backwards compatibility, Internet Explorer also provides the option to disable the usage of multiple processes and host all tabs in a single browser process.

\label{sec:brie}
The network stack used in Internet Explorer is provided by the Windows operating system and is called WinINet \cite{wininet}. This library provides high-level access to functions that perform HTTP and FTP requests and utility functions for caching, proxies and security. After initiating and configuring the request, WinINet will perform the necessary steps to execute the request. WinINet depends on the Winsock library \cite{winsock} to setup the required network connections and Schannel \cite{schannel} is used to provide transparent support for SSL/TLS connections.

\begin{figure}
    \centering
    \includegraphics[width=9cm]{Images/IE8_process_model.png}
    \caption{The Internet Explorer process model starting from version 8. \cite{IE8LCIEP}}
    \label{fig:ie8proc}
\end{figure}

\textbf{Firefox} uses only a single process for web content and only runs plug-ins from a different process. A long-term project to change that is called Electrolysis\footnote{https://wiki.mozilla.org/Electrolysis} and has been developed since 2009. In the new architecture, the entire rendering moved to a dedicated and sandboxed ``content'' process and the main process is used to host the user interface and serves as a proxy between the outside world and the content process. A longer-term goal is to spread the rendering of multiple tabs over more than one content process so when a content process crashes not all tabs are affected.

To be platform independent, Firefox does not directly interface with the provided libraries of the operating system. Instead a platform-neutral API called ``NSPR'' (Netscape Portable Runtime, \cite{nspr}) is used. Together with the Network Security Services (NSS, \cite{nss}) library that provides the functionality to create SSL/TLS connections, both are used by the high-level network library called Necko. Necko provides the interface to perform HTTP and other protocol requests without revealing the underlying protocol, transport level or platform specific implementation details and is thus comparable to WinINet.

\textbf{Chromium} is the open-source version of the Google Chrome browser and it is, except for a couple of proprietary components, identical to Chrome. The big innovation of Chrome \cite{ChromeMPA} was to use multiple processes instead of a single process. Besides its own process for every tab, it also has the plug-ins and audio subsystem in their respective processes. The subprocesses run in a sandbox with limited privileges and use the main process to communicate with the outside world.

\begin{figure}[h]
    \centering
    \includegraphics[width=9cm]{Images/Chrome_network.png}
    \caption{A high-level overview of the components involved in requesting a URL in Chromium. Platform specific details related to sockets are hidden in StreamSocket and the usage of SSL/TLS is made transparent by using the interface-compatible SSLSocket instead of StreamSocket. \cite{ChromeNetwork}}
    \label{fig:chrome_network}
\end{figure}

The library used by Chromium for network access is custom developed and tightly integrated in the engine. It provides similar functionality as Necko and WinINet, using a high-level interface (see figure \ref{fig:chrome_network}), but it also contains low-level interfaces that use the operating system's socket APIs directly. To provide transparent SSL/TLS support, the same library as Firefox is used, NSS.

\textbf{Safari} is the last browser that was examined in this project. Since 2011\footnote{https://lists.webkit.org/pipermail/webkit-help/2011-July/002298.html} support for using multiple processes has been added. Safari is closed-source but built on top of many open-source components like JavaScriptCore and WebKit.

Safari uses a dedicated process for every tab until a certain limit is reached. Once this limit is reached, multiple tabs are hosted in a single process. The network operations for the main and tab processes are concentrated in a dedicated process. Only this process will retrieve the webpages and uses the IPC subsystem to deliver the result to the correct process. 

CFNetwork is the library that is used by Safari for its network access. This library is one of the core frameworks of the OS X operating system and available for all applications. It provides interfaces for all relevant web related protocols. A unified interface called NSUrl, similar as seen in other libraries, is also available. However, because of the closed-source nature of Safari it is not possible, without an extensive reverse engineering effort, to determine if it is used instead of directly using the provided APIs of the CFNetwork library.

\iffalse

Which techniques are used by browsers to make concurrently visiting multiple
URLs possible?
	- Tabs of course
		- As a process
		- As multiple threads under the browser process

How can we link an HTTP request to its source URL without the modification of the used web browser?
we need extra information, see vraag 4
network is niet genoeg blablabla

How do web browsers make HTTP requests and retrieve webpages? Which Operating System level APIs are used?
	- Safari: CFNetwork and NSURLConnection(Loader) and IPC
		1 hoofdprocess: Safari
		1 process per tab: Safari Web Content
		1 process voor networking: Safari Networking
		Network stack of OS X.
	- Internet Explorer: C API calls naar Windows libraries
		Process per tab: is configureerbaar in settings/registry, je kan ook meerdere tabs in 1 process hebben
		Network stack of Windows
	- Firefox: C++ Library calls
		Single Process (+ 1 process voor flash)
		Own network stack Necko (nss3 voor trafiek te encrypten)
		\url{https://developer.mozilla.org/en-US/Firefox/Releases/3.5/Updating_extensions#Getting_a_load_context_from_a_request}
		http://stackoverflow.com/questions/10719606/is-it-possible-to-know-the-target-domwindow-for-an-httprequest
	- Chrome: IPC 
		1 hoofdprocess: Google Chrome die networking doet
		1 process per tab. (Altijd 13 threads?)
		1 process voor Flash.
		1 process voor Audio. (4 threads)
		Own network stack (nss3 voor trafiek te encrypten)
		http://www.chromium.org/developers/design-documents/network-stack

What extra information from the client's (running) machine can be used to augment the information gained from network trac to make the tracking of malware to its source URL easier?  
	- The Thread ID or Process ID van tabs, PDF reader, Java applet, ...
	- Handle bij IE
	- File descriptors
	- Process tree
	- Voordeel van op machine network traffic te intercepten is dat we rommel van andere applicaties niet zien, maar enkel het trafiek van de browser en de gespawnde subprocessen ervan.
	- 
\fi
