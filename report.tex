\documentclass{scrartcl}
\usepackage{multicol}
\usepackage{url}
\usepackage{listings}
\usepackage{graphicx}
\usepackage{color}
\usepackage[parfill]{parskip}
\usepackage{titlesec}
\usepackage{geometry}

\title{Large-scale drive-by download detection: boot. visit$^n$. kill. repeat.}
\subtitle{Master in System and Network Engineering}
\author{
  Adriaan Dens\\
   \texttt{adriaan.dens@os3.nl}
  \and
  Martijn Bogaard\\
   \texttt{martijn.bogaard@os3.nl}
}

\begin{document}
\maketitle

\tableofcontents

\section{Abstract}

\section{Acknowledgements}

\section{Introduction}

In the digital world of today, malware is still a massive and growing problem. While back in the day it was used to annoy users and system administrators, nowadays it's used for extortion, cyber espionage and surveillance by criminal groups and rivalling governments. One of the main risk factors to get infected with malware is a drive-by download while visiting a normal day-to-day website because, for example, the website got hacked and infected. 

In many cases\footnote{http://www.proofpoint.com/threatinsight/posts/malware-in-ad-networks-infects-visitors-and-jeopardizes-brands.php} \footnote{http://blog.fox-it.com/2013/08/01/analysis-of-malicious-advertisements-on-telegraaf-nl} \footnote{http://blog.fox-it.com/2014/01/03/malicious-advertisements-served-via-yahoo}, it was not the actual website but one of the advertisement networks that was infiltrated and which subsequently started serving malicious code hidden in innocently looking advertisement code. This is also called malvertising \cite{Li2012}.

National CERT organisations are (of course) interested in an early detection of such threats. While automated systems to scan websites already exist, like Cuckoo\footnote{http://cuckoosandbox.org} and Anubis\footnote{http://anubis.iseclab.org}, one of the main difficulties is the time needed to analyse a single website and the maintenance needed to keep these systems up-to-date for the latest threats.

% Zeggen dat _wij_ dit en dat gaan doen om die en dat redenen

\section{Scope}

% Aanpassen aangezien we weten wat we hebben gedaan
% Mergen in introductie sectie
The scope of this project consists of creating an algorithm that allows multiple URLs to be opened at the same time while still being able to track all further interaction, such as unexpected HTTP requests and other malicious activity, and link them to the original request/URL. To prove that the algorithm is something feasible, we will also implement a proof of concept of the algorithm.

The goal during this research project is to make the algorithm platform agnostic. If this is not possible then we will limit our self to Windows 7 with version 8 of the Internet Explorer browser.

The detection and identification of malicious behaviour is not part of this project. For our PoC we will stick to the detection of a well-known older and still to be determined malware family which existence is easy to detect on the system. 



\section{Research Question}
In cooperation with the Dutch National Cyber Security Center (NCSC-NL), our research project will focus on the question:\\

\textit{How can we concurrently visit multiple URLs and still be able to determine which URL was responsible for malicious activities?}\\
\newpage
To answer the research question, multiple subquestions have been formulated:
\begin{itemize}
\item Which techniques are used by browsers to make concurrently visiting multiple URLs possible?
\item How can we link an HTTP request to its source URL without the modification of the used web browser?
\item Which APIs are used by web browsers to make HTTP requests and retrieve webpages?
\item What extra information from the client's (running) machine can be used to augment the information gained from network traffic to make the tracking of malware to its source URL easier?
\end{itemize}




\section{Related work}
The growing importance and economic losses of malware resulted in many research projects in the last years. The detection and analysis of malware has been researched from several different angles \cite{auto_malware,Chang2013} and resulted in many proposed static and dynamic analysis techniques.

In 2013, Le et al. \cite{Le2013} presented a framework that describes the common stages and characteristics of a drive-by download attack. They described four stages from placing the malicious content on a webpage until the execution of the malicious activity.

In 2011, a paper from Canali et al. \cite{Canali2011} was released about the problematic performance of dynamic analysis and with a solution proposed in the form of ``Prophiler''. Prophiler is a filter that deploys static analysis techniques and that is able to reduce the load with more than 85\% compared to dynamic analysis. This with a comparable amount of false negatives.

In the same year Rajab et al. \cite{Rajab11trendsin} gave an overview of the trends regarding web malware detection and how the malware tries to circumvent the detection. This research focused on the advantages and disadvantages of four techniques: Virtual Machine honeypots, Browser Emulation honeypots, Classification based on Domain Reputation and Anti-Virus Engines.

A different approach is taken by Rossow \textit{et al.} \cite{Rossow2011}, Cortjens \textit{et al.} \cite{Cortjens2012} and Kinkhorst \textit{et al.} \cite{Kinkhorst2009}. They have focused during multiple research projects on the ability to detect and identify malware on the network layer.

The predecessor of NCSC-NL started in 2007 with the development of their own system, namely the Honeyspider network \cite{honeyspider}, for the dynamic analysis of websites. This system crawled the biggest and most important websites of the Netherlands on a daily base. The downside of this system is that it requires a lot of maintenance and hence it started to become outdated.




\section{Web browser internals}

Which techniques are used by browsers to make concurrently visiting multiple
URLs possible?
	- Tabs of course
		- As a process
		- As multiple threads under the browser process

How can we link an HTTP request to its source URL without the modification of the used web browser?
we need extra information, see vraag 4
network is niet genoeg blablabla

How do web browsers make HTTP requests and retrieve webpages? Which Operating System level APIs are used?
	- Safari: CFNetwork and NSURLConnection(Loader) and IPC
		1 hoofdprocess: Safari
		1 process per tab: Safari Web Content
		1 process voor networking: Safari Networking
		Network stack of OS X.
	- Internet Explorer: C API calls naar Windows libraries
		Process per tab: is configureerbaar in settings/registry, je kan ook meerdere tabs in 1 process hebben
		Network stack of Windows
	- Firefox: C++ Library calls
		Single Process (+ 1 process voor flash)
		Own network stack Necko (nss3 voor trafiek te encrypten)
		\url{https://developer.mozilla.org/en-US/Firefox/Releases/3.5/Updating_extensions#Getting_a_load_context_from_a_request}
		http://stackoverflow.com/questions/10719606/is-it-possible-to-know-the-target-domwindow-for-an-httprequest
	- Chrome: IPC 
		1 hoofdprocess: Google Chrome die networking doet
		1 process per tab. (Altijd 13 threads?)
		1 process voor Flash.
		1 process voor Audio. (4 threads)
		Own network stack (nss3 voor trafiek te encrypten)
		http://www.chromium.org/developers/design-documents/network-stack

What extra information from the client's (running) machine can be used to augment the information gained from network trac to make the tracking of malware to its source URL easier?  
	- The Thread ID or Process ID van tabs, PDF reader, Java applet, ...
	- Handle bij IE
	- File descriptors
	- Process tree
	- Voordeel van op machine network traffic te intercepten is dat we rommel van andere applicaties niet zien, maar enkel het trafiek van de browser en de gespawnde subprocessen ervan.
	- 

\section{Generic algorithm}

- Disconnected DAGs

1) Hooker schrijven voor elke browser die trafiek tussen Netwerk en Browser of in de browser zelf monitort en doorgeeft aan Cuckoo
2) DAG opstellen door de extra informatie te gebruiken in report.json
3) DAG interpreteren, nodes met enkel uitgaande edges zijn de beginnende URL
4) Kijken of er iets raar gebeurt in elk eiland
5) Reporting

\section{Platform specific problems}

Unix heeft geen handles zoals Windows maar meer files


\section{Proof of Concept}
	- Cuckoo aanpassen
		- Multiple URL support
		- Fixed Yara
	- Analyzer in Python
		- report.json parser
		- Commandline interface
			- Input: file, lijst op cli
			- Generic algorithm
			- Output: "No Malicious activity found", "URL xxxxx shows malicious activity: en dan wat korte informatie" 
		- 

\section{Results}

Speedboost tussen Cuckoo 1.2-dev en onze wijzigingen, grafiekjes maken


\section{Ethical issues}

% Zinnen herschrijven die naar de toekomst verwijzen.
Our research contains no major ethical issues as it does not include working with personally identifiable information. Malware, if any, will be run in a controlled virtual environment. After every testrun the virtual machine will be automatically destroyed.



\section{Conclusion}


\bibliographystyle{ieeetr}
\bibliography{report}

\end{document}
